\chapter{Conclusion}
We have discussed that the AGM framework introduces a reasonable set of rules that can be followed to reach rationality in building belief systems. However, we used a different formalism to express knowledge and perform contraction throughout this study than the one the AGM framework uses. This formalism, which is $\mathcal{EL}$, belongs to a family called Description Logic (DL). $\mathcal{EL}$ is one of the simple members of the DL family, and, as the rest of the DLs, uses concepts and roles as the building blocks of the knowledge base. We only showed how to perform contraction on TBoxes, and not ABoxes. We then discussed some basic approaches to contraction and introduced a general algorithm that can accommodate the use of different heuristics to seek optimality. 

We introduced a restricted contraction algorithm that uses graphs and solves contraction as a network flow problem. The graph approach is restricted in the sense that it can be used only in certain cases where inference in the TBox involves only the transitive property of the subsumption relation. We showed how the approach is sound and complete (only in the restricted cases) by showing that it follows most of the AGM postulates for contraction. We also discussed three heuristics, two of which are based on the semantics of the subsumption hierarchy of $\mathcal{EL}$, which are Localization and Specificity. We showed also how the greedy approach for contraction can be used.

The bottleneck for the contraction algorithms we discussed is actually generating all the kernels of a specific belief. The pinpointing algorithm we use for this purpose can sometimes take exponential number of steps with respect to the size of the TBox. Other than that, all the algorithms we discussed take polynomial time to run. 

What can be done in the future is to give an algorithm that contracts explicit knowledge in the ABox we well. Also, the use of $\mathcal{EL}$ was very important to finding polynomial time contraction algorithms. Would using other more expressive versions of DL result in higher order contraction than polynomial? What advantage would using a more expressive DL add and will it be worth it if contraction algorithms get harder and more complicated? 