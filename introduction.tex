\chapter{Introduction}

Learning is one of the most intuitive functions of the human mind. It involves translating human's perception into knowledge, and adapting the state of mind according to that knowledge. Seeing the sunrise in the morning, one would intuitively change his mind and believe that it is not dark anymore; that would logically imply that one gives up the current belief that it is dawn time and replace it with a new belief that it is sunrise time. This is an example of belief revision. Revision is a kind of change in which the new belief (``it is sunrise") is conflicting with the current state of mind (believing that ``it is dawn"). 

Learning can be thought of as the manipulation of beliefs according to perception. When humans perceive any change in their world, they change their knowledge accordingly. So changing the knowledge (or beliefs\footnote{Throughout this study we will be using the terms ``knowledge" and ``beliefs" interchangeably, although they are not exactly the same. Knowledge is usually assumed to be a special kind of beliefs. However, for convenience, when we use the word ``knowledge" we will be referring to ``belief." }) of an agent\footnote{The word ``agent" here means humans, computers, or any thing that has knowledge base and perception.} can be seen as the reality of learning. One example of belief change is \textbf{Revision}, which involves changing the knowledge base in order to add a conflicting belief. This can actually be broken down into two processes: changing the knowledge base to account for the conflict, and adding the new belief. 

The first process involves removing the beliefs that caused that conflict. This process is called \textbf{Belief Contraction}. The second process involves adding the new belief and expanding the knowledge base accordingly. This process is called \textbf{Belief Expansion}. Contraction is the type of change we are mainly concerned with in this study. In the following chapter, we will discuss in some level of detail these types of change with more focus on contraction.

The knowledge of an agent can be represented as a set of beliefs. An agent is assumed to believe in $A$ if $A$ is a member of its \textit{belief set}. In this study, the language used to represent belief sets is \textit{Description Logic}. Description Logic is a family of formalisms that are used to represent knowledge. They use concepts to represent classes of individuals and roles to represent relationships between them. They have different expressive powers and different reasoning mechanism with different complexities. They vary according to the set of logical operators they use. Here we use $\mathcal{EL}$, which is a member of the Description Logic family.

We start the study by discussing the basic types of belief change, but before that, we build a ground for them by defining the framework that was introduced by Gardenfors. We define epistemic states and attitudes that will help in understanding the mechanics of belief change. We then explain some postulates introduced in the AGM framework; those postulates are considered rationality rules for belief change operations. Then we discuss what description logic is, what logical operators are used and what it is composed of. In that discussion, we focus on the most relevant variation, which is $\mathcal{EL}$. And we explain the most important reasoning algorithm for $\mathcal{EL}$.